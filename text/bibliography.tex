\begin{thebibliography}{99}

\bibitem{first} Dempster A. P., Laird N. M., Rubin D. B. Maximum likelihood from incomplete data via the EM algorithm // J. of the Royal Statistical Society, Series B. — 1977. —no. 34. — Pp. 1–38.

\bibitem{second} Шлезингер М. И. О самопроизвольном различении образов // Читающие автоматы. — Киев, Наукова думка, 1965. — Pp. 38–45

\bibitem{third} Айвазян С. А., Бухштабер В. М., Енюков И. С., Мешалкин Л. Д. Прикладная статистика: классификация и снижение размерности. — М.: Финансы и статистика, 19

\bibitem{fourth} В. Ю. Королёв. ЕМ-алгоритм, его модификации и их применение к задаче разделения смесей вероятностных рспределений

\bibitem{fifth} К. В. Воронцов. Лекции по статистическим (байесовским) алгоритмам классификации

\bibitem{eighth} Wu C. F. G. On the convergence properties of the EM algorithm // The Annals of
Statistics. — 1983. — no. 11. — Pp. 95–103.

\bibitem{sixth} Документация GaussianMixture \\ \url{https://scikit-learn.org/stable/modules/generated/sklearn.mixture.GaussianMixture.html#sklearn.mixture.GaussianMixture}

\bibitem{seventh} Документация к модулю stats библиотеки Scipy \\ \url{https://docs.scipy.org/doc/scipy/reference/tutorial/stats.html}

\bibitem{nineth} GitHub-репозиторий выпускной квалификационной работы \\ \url{https://github.com/ruslanmuradasilov/diploma}

\end{thebibliography}