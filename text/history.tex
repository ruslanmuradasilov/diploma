Одним из первых EM-алгоритм был предложен McKendrick (1926) для медицинских приложений. Затем после довольно большого перерыва эта идея вновь возникла в работах Healy and Westmacott (1956), Шлезингера (1965, 1968), Day (1969), Wolfe (1970), а затем развита  и систематически исследована в работе Dempster, Laird and Rubin (1977) \cite{first}. Само название \emph{EM-алгоритм} было предложено в работе \cite{first}, в которой показана высокая общность алгоритма. Возможно, поэтому зарубежные источники традиционно ссылаются на эту статью, как на первую работу по EM-алгоритму.

Основные свойства ЕМ-алгоритма описаны еще в работе Шлезингера (1965) \cite{second}. Позднее в работах Dempster, Laird and Rubin (1977), Everitt and Hand (1981), Wu (1983), Boyles (1983), Redner and Walker (1984) эти свойства были передоказаны и развиты.

Литература по ЕМ-алгоритму и его применениям к решению задач из конкретных областей обширна. Среди них можно выделить книги, посвященные собственно ЕМ-алгоритму Литтла и Рубина (1991), McLachlan and Krishnan (1997), книги, в которых ЕМ-алгоритму уделено значительное место Айвазяна и др. (1989) \cite{third}, Tanner (1993), а также двух обстоятельных работ Bilmes (1998) и Figueiredo (2004).