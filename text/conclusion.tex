    В данной работе было собрано методическое пособие по знакомству с EM-алгоритмом на примере смеси гауссовых распределений. Было дано описание самого алгоритма, которое позволит быстро ознакомиться с теоритической частью EM-алгоритма.
    
    Также был выполнен следующий прикладной вклад. Во-первых, было наглядно продемонстрирована генерация выборок разных распределений и исследованы результаты способностей GaussianMixture оценивать параметры таких выборок. Во-вторых, был сделан новый шаг по вычислению математического ожидания и дисперсии цензурированных выборок на основе оценки Абдушукурова А. А. и предложен способ по применению EM-алгоритма к таким выборкам на реальных данных.
    
    В дальнейшем предполагается увеличение компонент смеси распределений и расширение семейства самих распределений, подаваемых на вход EM-алгоритму. К примеру, разработка GeneralMixture, позволяющей оценивать параметры любого семейства распределений.
    
    Автор выражает благодарность А. А. Абдушукурову за научное руководство.