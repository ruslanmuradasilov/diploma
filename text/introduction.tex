EM-алгоритм в математической статистике используется для нахождения оценок максимального правдоподобия параметров вероятностной модели, в случае, когда модель зависит от некоторых скрытых данных. Как правило, EM-алгоритм применяется при решении задач двух типов.

К первому типу относятся задачи с \emph{действительно} неполными данными, когда некоторые статистические данные отсутствуют в силу каких-либо причин. Ко второму же типу можно отнести задачи, в которых удобно вводить скрытые переменные для упрощения подсчета функции правдоподобия. Примером такой задачи может служить кластеризация.

В данной работе приведено описание EM-алгоритма и его свойства, а также предложен пример с его применением. 