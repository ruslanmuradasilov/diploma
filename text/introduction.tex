EM-алгоритм в математической статистике используется для нахождения оценок максимального правдоподобия параметров вероятностной модели, в случае, когда модель зависит от некоторых скрытых данных. Как правило, EM-алгоритм применяется при решении задач двух типов.

К первому типу относятся задачи с \emph{действительно} неполными данными, когда некоторые статистические данные отсутствуют в силу каких-либо причин. Ко второму же типу можно отнести задачи, в которых удобно вводить скрытые переменные для упрощения подсчета функции правдоподобия. Примером такой задачи может служить кластеризация.

В данной работе приведено описание EM-алгоритма и его свойства, а также предложен пример с его применением к задаче первого типа.

В первой части работы дан теоретический материал, в котором подробно изложены E-шаг и M-шаг в общем случае, а также на примере разделения смеси нормальных распределений.

Вторая часть работы посвящена применению EM-алгоритма с использованием модуля GaussianMixture библиотеки Scikit-Learn на языке Python. В первую очередь были сгенерированы разные выборки с помощью модуля Stats библиотеки SciPy, на основе которых была проведена проверка работоспособности и степень точности оценки алгоритма на данных разных распределений. Далее алгоритм был применен к цензурированной выборке, предварительно предобратонной с помощью оценки $F_n^{RR}$, предложенной Абдушукуровым А. А.